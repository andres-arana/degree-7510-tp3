\documentclass[a4paper,11pt]{article}

%%%%%%%%%%%%%%%%%%%%%%%%%%%%%%%%%%%%%%%%%%%%%%%%%%%%%%%%%%%%%%%%%%%%%%%%
% Paquetes utilizados
%%%%%%%%%%%%%%%%%%%%%%%%%%%%%%%%%%%%%%%%%%%%%%%%%%%%%%%%%%%%%%%%%%%%%%%%

% Gráficos complejos
\usepackage{graphicx}
\usepackage{caption}
\usepackage{subcaption}
\usepackage{placeins}

% Soporte para el lenguaje español
\usepackage{textcomp}
\usepackage[utf8]{inputenc}
\usepackage[T1]{fontenc}
\DeclareUnicodeCharacter{B0}{\textdegree}
\usepackage[spanish]{babel}

% Formato de párrafo
\setlength{\parskip}{1ex plus 0.5ex minus 0.2ex}

%%%%%%%%%%%%%%%%%%%%%%%%%%%%%%%%%%%%%%%%%%%%%%%%%%%%%%%%%%%%%%%%%%%%%%%%
% Título
%%%%%%%%%%%%%%%%%%%%%%%%%%%%%%%%%%%%%%%%%%%%%%%%%%%%%%%%%%%%%%%%%%%%%%%%

% Título principal del documento.
\title{\textbf{Trabajo Práctico N°3}}

% Información sobre los autores.
\author{
  Andrés Gastón Arana(and2arana@gmail.com), \textit{P. 86.203}     \\
  Gabriel Ostrowsky(gaby.ostro@gmail.com), \textit{P. 90.762}       \\
  Ignacio Garay Ojeda(imgarayojeda@gmail.com), \textit{P. 92.265}   \\
  Pablo Angelani(pablo.angelani@gmail.com), \textit{P. 92.707}      \\
  \\
  \normalsize{1er. Cuatrimestre de 2013}                           \\
  \normalsize{75.10 - Técnicas de Diseño}                          \\
  \normalsize{Facultad de Ingeniería, Universidad de Buenos Aires}
}
\date{}

%%%%%%%%%%%%%%%%%%%%%%%%%%%%%%%%%%%%%%%%%%%%%%%%%%%%%%%%%%%%%%%%%%%%%%%%
% Documento
%%%%%%%%%%%%%%%%%%%%%%%%%%%%%%%%%%%%%%%%%%%%%%%%%%%%%%%%%%%%%%%%%%%%%%%%

\begin{document}

% ----------------------------------------------------------------------
% Top matter
% ----------------------------------------------------------------------
\thispagestyle{empty}
\maketitle

\begin{abstract}

  Este informe sumariza el desarrollo del trabajo práctico grupal N°3 de la
  materia Técnicas de Diseño (75.10) dictada en el primer cuatrimestre de 2013
  en la Facultad de Ingeniería de la Universidad de Buenos Aires. El mismo
  consiste en la extensión del trabajo práctico número 2 para incorporar
  conceptos de LRP a través de la asignación de puntos a clientes para ser
  canjeados en posteriores ventas. El trabajo práctico a extender es el
  producido para la segunda entrega de la materia por otro grupo de trabajos
  prácticos.

\end{abstract}

\clearpage

% ----------------------------------------------------------------------
% Tabla de contenidos
% ----------------------------------------------------------------------
\tableofcontents
\clearpage


% ----------------------------------------------------------------------
% Desarrollo
% ----------------------------------------------------------------------
\part{Desarrollo}

En esta sección se realiza un análisis del diseño que exhibe el trabajo
práctico a extender, remarcando ventajas y desventajas comparativas con el
desarrollado por nuestro propio grupo. En función de estas se determinan
alternativas de implementación para los requisitos nuevos dados.

Se detalla la alternativa seleccionada y los motivos por los que se realizó la
elección. Posteriormente se realiza un análisis de las dificultades encontradas
durante la implementación de las extensiones al diseño.

\section{Análisis comparativo de diseño}

\subsection{Puntos positivos}

\subsection{Críticas}

\section{Extensión del diseño}

\subsection{Alternativas de diseño}

\subsection{Dificultades en la implementación}

\end{document}

