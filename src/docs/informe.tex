\documentclass[a4paper,11pt]{article}

%%%%%%%%%%%%%%%%%%%%%%%%%%%%%%%%%%%%%%%%%%%%%%%%%%%%%%%%%%%%%%%%%%%%%%%%
% Paquetes utilizados
%%%%%%%%%%%%%%%%%%%%%%%%%%%%%%%%%%%%%%%%%%%%%%%%%%%%%%%%%%%%%%%%%%%%%%%%

% Gráficos complejos
\usepackage{graphicx}
\usepackage{caption}
\usepackage{subcaption}
\usepackage{placeins}

% Soporte para el lenguaje español
\usepackage{textcomp}
\usepackage[utf8]{inputenc}
\usepackage[T1]{fontenc}
\DeclareUnicodeCharacter{B0}{\textdegree}
\usepackage[spanish]{babel}

% Formato de párrafo
\setlength{\parskip}{1ex plus 0.5ex minus 0.2ex}

%%%%%%%%%%%%%%%%%%%%%%%%%%%%%%%%%%%%%%%%%%%%%%%%%%%%%%%%%%%%%%%%%%%%%%%%
% Título
%%%%%%%%%%%%%%%%%%%%%%%%%%%%%%%%%%%%%%%%%%%%%%%%%%%%%%%%%%%%%%%%%%%%%%%%

% Título principal del documento.
\title{\textbf{Trabajo Práctico N°3}}

% Información sobre los autores.
\author{
  Andrés Gastón Arana(and2arana@gmail.com), \textit{P. 86.203}     \\
  Gabriel Ostrowsky(gaby.ostro@gmail.com), \textit{P. 90.762}       \\
  Ignacio Garay Ojeda(imgarayojeda@gmail.com), \textit{P. 92.265}   \\
  Pablo Angelani(pablo.angelani@gmail.com), \textit{P. 92.707}      \\
  \\
  \normalsize{1er. Cuatrimestre de 2013}                           \\
  \normalsize{75.10 - Técnicas de Diseño}                          \\
  \normalsize{Facultad de Ingeniería, Universidad de Buenos Aires}
}
\date{}

%%%%%%%%%%%%%%%%%%%%%%%%%%%%%%%%%%%%%%%%%%%%%%%%%%%%%%%%%%%%%%%%%%%%%%%%
% Documento
%%%%%%%%%%%%%%%%%%%%%%%%%%%%%%%%%%%%%%%%%%%%%%%%%%%%%%%%%%%%%%%%%%%%%%%%

\begin{document}

% ----------------------------------------------------------------------
% Top matter
% ----------------------------------------------------------------------
\thispagestyle{empty}
\maketitle

\begin{abstract}

  Este informe sumariza el desarrollo del trabajo práctico grupal N°3 de la
  materia Técnicas de Diseño (75.10) dictada en el primer cuatrimestre de 2013
  en la Facultad de Ingeniería de la Universidad de Buenos Aires. El mismo
  consiste en la extensión del trabajo práctico número 2 para incorporar
  conceptos de LRP a través de la asignación de puntos a clientes para ser
  canjeados en posteriores ventas. El trabajo práctico a extender es el
  producido para la segunda entrega de la materia por otro grupo de trabajos
  prácticos.

\end{abstract}

\clearpage

% ----------------------------------------------------------------------
% Tabla de contenidos
% ----------------------------------------------------------------------
\tableofcontents
\clearpage


% ----------------------------------------------------------------------
% Desarrollo
% ----------------------------------------------------------------------
\part{Desarrollo}

En esta sección se realiza un análisis del diseño que exhibe el trabajo
práctico a extender, remarcando ventajas y desventajas comparativas con el
desarrollado por nuestro propio grupo. En función de estas se determinan
alternativas de implementación para los requisitos nuevos dados.

Se detalla la alternativa seleccionada y los motivos por los que se realizó la
elección. Posteriormente se realiza un análisis de las dificultades encontradas
durante la implementación de las extensiones al diseño.

\section{Análisis comparativo de diseño}

Luego de recibir el paquete de trabajo del otro grupo se comenzó por revisar
los diagramas de diseño para entender el panorama amplio de la solución y
realizar un análisis comparativo con nuestro diseño original. El objetivo era
identificar los puntos concretos que difieren y así realizar un posterior
análisis de bajo nivel en esa sección del código.

A continuación se detallan los puntos positivos y negativos en relación a
nuestra implementación original, según extensibilidad y capacidad de resolución
del problema.

\subsection{Puntos positivos}

El trabajo recibido tenía algunos puntos positivos interesantes que
se detallan a continuación:

\begin{enumerate}

  \item Se incluye un API para construir ofertas de una manera más amigable y
    dinámica, usando el patrón builder. En nuestro diseño el proceso de armado
    de ofertas es algo trabajoso, debiendo componer las condiciones y
    descuentos de forma manual.

  \item Se desarrollo el concepto de ofertas recurrentes y no recurrentes. Si
    bien el nombre que identifica el concepto de recurrencia no es
    representativo de lo que significa y la implementación no es ideal, la
    lógica de aplicar la oferta asociada hasta que la condición sea falsa es
    interesante y constituye un concepto poderoso en la implementación de
    ofertas exclusivas.

  \item Se incluyen diagramas muchos más claros y con formato de estilo para
    una mejor comprensión, uno de los puntos más débiles de nuestra entrega

\end{enumerate}

\subsection{Críticas}

Se encontraron algunos puntos que se podrían mejorar en la implementación
recibida, dentro del análisis comparativo con respecto a nuestra entrega que se
realizó, detallados a continuación:

\begin{enumerate}

  \item Si bien el diseño existente separa la condición de aplicabilidad de las
    ofertas de la acción a ejecutar cuando la oferta es aplicada, consideramos
    que el nivel de granularidad de estas responsabilidades es demasiado grueso
    para poder implementar el esquema de puntos reutilizando los bloques ya
    desarrollados. Se decidió continuar trabajando en el nivel de granularidad
    ya definido, aunque sería bueno hacer más granular la composición de las
    condiciones y de las acciones de las ofertas en un refactor posterior para
    poder tener mayor combinatoria al implementar las nuevas ofertas.

  \item Sería necesaria una lógica más genérica de marcar productos para
    ofertas exclusivas. En este momento, cada oferta que es exclusiva con otras
    requiere que se modifique el producto para que almacene una nueva cantidad
    de aplicación a la oferta. Por ejemplo, existe un campo que representa la
    cantidad de ofertas que tiene aplicadas por producto, otro que representa
    el mismo concepto pero por marca y categoría, etc. Sería interesante
    agregar un esquema más genérico de categorización de estas cantidades, en
    un refactor posterior.

  \item Existen abstracciones innecesarias que dificultan la navegabilidad del
    código, la cual además puede prestar a confusión, como por ejemplo IVenta -
    Venta. Consideramos que al existir la interfaz, confunde la usabilidad del
    código, sugiriendo que puede existir otra implementación de venta distinta
    cuando las reglas de negocio no sugieren la necesidad de una extensión de
    esta funcionalidad en un futuro. Más aún, cada cambio a Venta implica un
    cambio adicional en la interfaz, lo que genera un costo adicional en el
    mantenimiento del código.
  
  \item Encontramos que la implementación no estaba del todo limpia y
    organizada. Existe un acoplamiento entre las responsabilidades relacionadas
    a la presentación y al modelo, como también salidas por pantalla propios de
    la presentación, en el código del modelo en varios lugares lados, los
    cuales salían incluso al momento de correr los tests.

  \item La calidad de código es algo pobre: Métodos largos y difíciles de
    entender; no existe documentación dentro del código (sea javadoc o código
    auto-documentado); no hay convenciones consistentes del estilo de código;
    hay variables y métodos con nombres por demás de largos y procedurales.

  \item En relación con la calidad de código, no se ve un buen uso y
    aprovechamiento de las convenciones e idioms propios del lenguaje (Java)
    elegido: no se respetan las convenciones de nombres generales del lenguaje;
    no se utilizan idioms como inmutabilidad a través de final, entre otros.

\end{enumerate}

\section{Extensión del diseño}

Se debía extender el diseño para incorporar el concepto de LRP. La idea general
es permitir mecanismos que asignen puntos a los clientes registrados del
mercado por comprar ciertos productos o por abonar montos determinados. Estos
puntos son intransferibles entre clientes, y pueden ser utilizados para pagar
una posterior compra, en relación un punto equivalente a un peso.

Para implementar estos requisitos se analizaron dos alternativas de diseño que
permitiesen incorporar la nueva lógica de negocio. A continuación se detallan
las alternativas propuestas y se define con más profundidad la elegida e
implementada.

\subsection{Alternativas de diseño}

Se identificaron dos alternativas de diseño válidas, cada una con sus tradeoffs
de calidad y tiempo de implementación, que se podían utilizar para implementar
los requisitos nuevos:

\begin{enumerate}

  \item Desarrollar un subsistema paralelo, independiente, de asignación de
    puntos.

  \item Incorporar nuevas capacidades al sistema existente de ofertas para
    asignar puntos.

\end{enumerate}

La segunda alternativa se debe a la similitud entre el funcionamiento de los
nuevos requisitos con respecto a lo ofrecido por el sistema de ofertas
existente. En ambos casos, dadas ciertas condiciones, se calcula un valor que
se aplica a la venta en forma de descuento. La diferencia radica en que la
oferta es inmediata, el valor es descontado en la compra en la que aplica la
misma, mientras que los puntos acumulados son aplicables a compras posteriores
de manera no inmediata.

Inicialmente, la primer alternativa parece más rápida y segura de implementar
que la segunda. La posibilidad de implementar la nueva lógica de negocio en un
subsistema separado es atractiva, dado que no es necesario verificar que no se
estén rompiendo casos válidos preexistentes en el sistema de ofertas. Por otro
lado, se puede argumentar que constituye una solución de calidad inferior, dado
que se estaría replicando la lógica común a ambas funcionalidades en dos
subsistemas independientes.

Después de analizar en profundidad el esquema de implementación actual del
sistema de ofertas, se decidió extender el mismo para incorporar los conceptos
de acumulación de puntos, lo que corresponde a la segunda alternativa
presentada.

\subsection{Implementación}

Para implementar la alternativa elegida, se construyó una nueva clase Cliente,
que representa la cuenta corriente de puntos de un cliente particular. Se
diseñó además un el correspondiente repositorio, para permitir en un futuro la
persistencia de estas entidades. Se modificó la Venta para que acepté un
Cliente, opcional, al momento de iniciarse, de manera de que esta pueda
registrar los puntos que se acumulan con la compra en la cuenta corriente del
cliente.

Finalmente, se desarrollaron nuevas implementaciones de condiciones y acciones
de ofertas para representar ofertas que asignan puntos a la cuenta corriente
del cliente. Estas ofertas se almacenan y procesan secuencialmente junto con el
resto de las ofertas de descuentos, pero las acciones asociadas aplican los
puntos correspondientes a la venta en la cuenta corriente de puntos del cliente
asociado a la venta.

\subsection{Dificultades en la implementación}

Durante la implementación del diseño elegido presentado en la sección anterior,
se encontraron algunos problemas de integración menores que fueron solucionados
a medida que se avanzaba con la misma. Uno de los problemas reflejaba una
decisión de diseño que tuvo que ser cambiada para acomodar la nueva
implementación, y deja entrever una problemática de diseño que podría ser
refactorizada en modificaciones posteriores. La misma se describe a
continuación.

Existen problemas con respecto al momento en el que se aplican las ofertas.
Anteriormente se aplicaban cada vez que se agregaba un producto a la venta.  El
problema de este criterio radica en que se reaplicaban algunas ofertas cada vez
que se agregaba un nuevo producto. Para solucionarlo, el diseño existente
incluye algunos workarounds; por ejemplo, las ofertas de descuentos
porcentuales globales acumulan un porcentaje de descuento en la venta que se
aplica al final únicamente, en vez de agregar registros de descuentos
individuales como lo hacen el resto de las ofertas. Más aún, no existe la
posibilidad de quitar items una vez agregados, dado que esto implicaría
identificar qué descuentos están aplicados a dichos productos, algo que es
complicado con el diseño actual.

Se resolvió por quitar la lógica que reaplica los descuentos al agregar cada
producto, y diferir el cálculo al momento en el que se confirma la venta.

\end{document}

