\documentclass[a4paper,11pt]{article}

%%%%%%%%%%%%%%%%%%%%%%%%%%%%%%%%%%%%%%%%%%%%%%%%%%%%%%%%%%%%%%%%%%%%%%%%
% Paquetes utilizados
%%%%%%%%%%%%%%%%%%%%%%%%%%%%%%%%%%%%%%%%%%%%%%%%%%%%%%%%%%%%%%%%%%%%%%%%

% Gráficos complejos
\usepackage{graphicx}
\usepackage{caption}
\usepackage{subcaption}
\usepackage{placeins}

% Soporte para el lenguaje español
\usepackage{textcomp}
\usepackage[utf8]{inputenc}
\usepackage[T1]{fontenc}
\DeclareUnicodeCharacter{B0}{\textdegree}
\usepackage[spanish]{babel}

% Formato de párrafo
\setlength{\parskip}{1ex plus 0.5ex minus 0.2ex}

%%%%%%%%%%%%%%%%%%%%%%%%%%%%%%%%%%%%%%%%%%%%%%%%%%%%%%%%%%%%%%%%%%%%%%%%
% Título
%%%%%%%%%%%%%%%%%%%%%%%%%%%%%%%%%%%%%%%%%%%%%%%%%%%%%%%%%%%%%%%%%%%%%%%%

% Título principal del documento.
\title{\textbf{Trabajo Práctico N°3}}

% Información sobre los autores.
\author{
  Andrés Gastón Arana(and2arana@gmail.com), \textit{P. 86.203}     \\
  Gabriel Ostrowsky(gaby.ostro@gmail.com), \textit{P. 90.762}       \\
  Ignacio Garay Ojeda(imgarayojeda@gmail.com), \textit{P. 92.265}   \\
  Pablo Angelani(pablo.angelani@gmail.com), \textit{P. 92.707}      \\
  \\
  \normalsize{1er. Cuatrimestre de 2013}                           \\
  \normalsize{75.10 - Técnicas de Diseño}                          \\
  \normalsize{Facultad de Ingeniería, Universidad de Buenos Aires}
}
\date{}

%%%%%%%%%%%%%%%%%%%%%%%%%%%%%%%%%%%%%%%%%%%%%%%%%%%%%%%%%%%%%%%%%%%%%%%%
% Documento
%%%%%%%%%%%%%%%%%%%%%%%%%%%%%%%%%%%%%%%%%%%%%%%%%%%%%%%%%%%%%%%%%%%%%%%%

\begin{document}

% ----------------------------------------------------------------------
% Top matter
% ----------------------------------------------------------------------
\thispagestyle{empty}
\maketitle

\begin{abstract}

  Este informe sumariza el desarrollo del trabajo práctico grupal N°3 de la
  materia Técnicas de Diseño (75.10) dictada en el primer cuatrimestre de 2013
  en la Facultad de Ingeniería de la Universidad de Buenos Aires. El mismo
  consiste en la extensión del trabajo práctico número 2 para incorporar
  conceptos de LRP a través de la asignación de puntos a clientes para ser
  canjeados en posteriores ventas. El trabajo práctico a extender es el
  producido para la segunda entrega de la materia por otro grupo de trabajos
  prácticos.

\end{abstract}

\clearpage

% ----------------------------------------------------------------------
% Tabla de contenidos
% ----------------------------------------------------------------------
\tableofcontents
\clearpage


% ----------------------------------------------------------------------
% Desarrollo
% ----------------------------------------------------------------------
\part{Desarrollo}

En esta sección se realiza un análisis del diseño que exhibe el trabajo
práctico a extender, remarcando ventajas y desventajas comparativas con el
desarrollado por nuestro propio grupo. En función de estas se determinan
alternativas de implementación para los requisitos nuevos dados.

Se detalla la alternativa seleccionada y los motivos por los que se realizó la
elección. Posteriormente se realiza un análisis de las dificultades encontradas
durante la implementación de las extensiones al diseño.

\section{Análisis comparativo de diseño}

Luego de recibir el paquete de trabajo del otro grupo, comenzamos por observar 
los Diagramas de Diseño, entendiendo el panorama amplio de la solución, a
la vez de realizar un análisis comparativo con nuestro diseño original, 
para identificar los puntos concretos que difieren y así realizar un posterior 
análisis de bajo nivel en esa parte del código.
Además, utilizamos unos minutos para utilizar la aplicación y reconocer las 
distintas funcionalidades y su forma de uso, lo que nos ayudó a entender mejor 
las responsabilidades e interacciones de los distintos objetos.
A continuación se detallan los puntos positivos y negativos en 
relación a nuestra implementación original, según extensibilidad y resolución 
del problema.

\subsection{Puntos positivos}

  * API para construir ofertas de una manera más amigable y dinámica, usando el patrón
  builder. Este diseño da una mejor experiencia de usuario pudiendo modificar la 
  información sin cerrar el programa ni editar el programa. Nuestro TP en cambio, 
  tiene un armado desacoplado pero estático en tiempo de compilación.

  * Ofertas recurrentes y no recurrentes. Si bien el nombre no es
  representativo de lo que hacen y la implementación no es ideal, la lógica
  de reaplicar la oferta asociada hasta que la condicion sea falsa es interesante
  y constituye un concepto poderoso en la implementación de ofertas exclusivas.
  
  * Diagramas claros y con formato de estilo para una mejor comprensión.

\subsection{Críticas}

[TODO: Agregar cosas que no están tan buenas del diseño del tp que nos toco. Algunas ideas:
  * Mala separación de responsabilidades, en particular en lo relacionado a
  la presentación y al modelo. Prints en el código del modelo por todos
  lados que salían incluso al momento de correr los tests.

  * Si bien el diseño existente separa la condición de aplicabilidad de las
  ofertas de la acción a ejecutar cuando la oferta es aplicada,
  consideramos el nivel de granularidad de estas responsabilidades es demasiado
  grueso para poder implementar el esquema de puntos reutilizando los bloques ya
  desarrollados. Decidimos continuar trabajando en el nivel de granularidad ya
  definido, aunque sería bueno hacer más granular la composición de las
  condiciones y de las acciones de las ofertas en un refactor posterior para
  poder tener mayor combinatoria al implementar las nuevas ofertas.

  * Sería necesaria una lógica más genérica de marcar productos para ofertas
  exclusivas. En este momento, cada oferta que es exclusiva con otras
  requiere que se modifique el producto para que almacene una nueva cantidad de
  aplicación (por ejemplo, cantidad que tiene aplicadas ofertas por producto,
  cantidad que tiene aplicadas ofertas por marca y categoría, etc.). Sería
  interesante agregar un esquema más genérico de categorización de estas
  cantidades.

  * Abstracciones por el sólo hecho de tener una indirección, complicando la
  navegabilidad del código sin motivo aparente. Ej: IVenta-Venta, qué motivo
  hay para variar la implementación de la venta cuando hay una sóla? Esto es
  discutible igual, pero me parece que abstraer porque sí complica el código al
  pedo.

  * Se podría mejorar algo la calidad de código. Algunos métodos son largos y
  difíciles de entender, no existe documentación dentro del código (sea javadoc
  o código autodocumentado), no hay convenciones claras del estilo de código,
  variables y métodos con nombres super-largos y procedurales, etc.  etc.

  * Relacionado con lo anterior. No se siguen las convenciones ni idioms
  propios de Java, el lenguaje que eligieron para implementar el trabajo
  práctico. No se respetan las convenciones de nombres generales del lenguaje,
  no se utilizan idioms como inmutabilidad a través de final, etc.

]

\section{Extensión del diseño}

[TODO: Intro a los nuevos requisitos.]

\subsection{Alternativas de diseño}

[TODO: Describir las dos alternativas: sistema paralelo o adaptar el sistema de
ofertas. Concluir con que decidimos adaptar el sistema de ofertas.]

\subsection{Dificultades en la implementación}

[TODO: Describir problemas que tuvimos:

  * Problemas con el momento en el que se aplican las ofertas. Se aplicaban
  cada vez que se agregaba un producto. El problema estaba en que se
  reaplicaban algunas ofertas. Para solucionarlo, el diseño incluye algunos
  hacks medios feos: por ejemplo, las ofertas globales acumulan un porcentaje
  de descuento que se aplica al final únicamente, en vez de agregar registros
  de descuentos individuales a la venta (como lo hacen el resto de las ofertas
  locales).

]

\end{document}

